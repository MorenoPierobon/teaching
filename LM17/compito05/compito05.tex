\documentclass[10pt]{article}
\usepackage[utf8]{inputenc}
\usepackage[a4paper,height=24cm,width=13cm]{geometry}
\usepackage[italian]{babel}
\usepackage{amssymb}
\usepackage{dsfont}
\usepackage{calc}
\usepackage{graphicx}
\usepackage{pstricks}
\usepackage{pst-node}
\usepackage{fourier}
\usepackage{euscript}
\usepackage{amsmath,amssymb, amsthm}

\def\lh{\textrm{lh}}
\def\phi{\varphi}
\def\P{\EuScript P}
\def\M{\EuScript M}
\def\D{\EuScript D}
\def\U{\EuScript U}
\def\S{\EuScript S}
\def\sm{\smallsetminus}
\def\niff{\nleftrightarrow}
\def\ZZ{\mathds Z}
\def\NN{\mathds N}
\def\PP{\mathds P}
\def\QQ{\mathds Q}
\def\RR{\mathds R}
\def\<{\langle}
\def\>{\rangle}
\def\E{\exists}
\def\A{\forall}
\def\0{\varnothing}
\def\imp{\rightarrow}
\def\iff{\leftrightarrow}
\def\IMP{\Rightarrow}
\def\IFF{\Leftrightarrow}
\def\range{\textrm{im}}
\def\Mod{\textrm{Mod}}
\def\Aut{\textrm{Aut}}
\def\Th{\textrm{Th}}
\def\acl{\textrm{acl}}
\def\eq{{\rm eq}}
\def\tp{\textrm{tp}}
\def\equivL{\stackrel{\smash{\scalebox{.5}{\rm L}}}{\equiv}}
\def\swedge{\mathbin{\raisebox{.2ex}{\tiny$\mathbin\wedge$}}}
\def\svee{\mathbin{\raisebox{.2ex}{\tiny$\mathbin\vee$}}}

\newcommand{\labella}[1]{{\sf\footnotesize #1}\hfill}
\renewenvironment{itemize}
  {\begin{list}{$\triangleright$}{%
   \setlength{\parskip}{0mm}
   \setlength{\topsep}{0mm}
   \setlength{\rightmargin}{0mm}
   \setlength{\listparindent}{0mm}
   \setlength{\itemindent}{0mm}
   \setlength{\labelwidth}{3ex}
   \setlength{\itemsep}{0mm}
   \setlength{\parsep}{0mm}
   \setlength{\partopsep}{0mm}
   \setlength{\labelsep}{1ex}
   \setlength{\leftmargin}{\labelwidth+\labelsep}
   \let\makelabel\labella}}{%
   \end{list}}
%\def\ssf#1{\textsf{\small #1}}
\newcounter{ex}
\newenvironment{exercise}{\clearpage\addtocounter{ex}{1}\textbf{Esercizio \theex.\quad}}{}
\pagestyle{empty}
\parindent0ex
\parskip2ex
\raggedbottom
\def\nsR{{}^*\!\RR}
\def\ssf#1{\textsf{#1}}
\renewcommand{\baselinestretch}{1.3}


\usepackage{fancyhdr}
\pagestyle{fancy}
\lhead{Logica Matematica a.a.~2016/17\hfill Nome Cognome}
\rhead{}
\cfoot{}
\rfoot{\rput(1.5,-0.5){\small\thepage}}

\begin{document}


\clearpage%%%%%%%%%%%%%%%%%%%%%%%%%%%%%%%
\rhead{}\setcounter{ex}{0}


\begin{exercise}
Prove that for every $\Delta$-type $p$ the following are equivalent
\begin{itemize}
\item[1.] $p$ is prime;
\item[2.] $\displaystyle p\ \vdash\ \bigvee^n_{i=1}\phi_i$\parbox{6ex}{\hfil$\IMP$}$p\vdash\phi_i$ for some $i\le n$,\hfill for every $n$ and every $\phi_1,\dots,\phi_n\in\Delta$.
\end{itemize}

(È suffciente una frase che spieghi da cosa segue l'equivalenza.)
\end{exercise}



\begin{exercise}
Let $\PP$ be an upper semilattice. Let $B\subseteq \PP$ and let $c\in \PP$ be such that $\swedge C\not\le c$ for every finite non-empty $C\subseteq B$. Prove that the following are equivalent
\begin{itemize}
\item[1.] $B$ is a maximal filter relative to $c$;
\item[2.] $a\notin B\ \ \IMP\ \ b\swedge a \le c$\ \  for some \ \ $b\in B$.
%\item[3.] $a\notin B\ \ \IMP\ \ \swedge C\swedge a\le c$\ \ per qualche $C\subseteq B$ finito non vuoto.
\end{itemize}
\end{exercise}


\begin{exercise}
Let $\PP$ be an upper semilattice. Let $F\subseteq\PP$ be a principal filter. Is $F$ always contained in a maximal principal filter\,?

(Non serve dimostrazione dettagliata, basta un controesempio.)
\end{exercise}


\begin{exercise}
Let $\PP$ be a distributive lattice. Suppose we defined $S(\PP)$ as the set of relatively maximal filters. Which essential (for Stone duality) property would not hold\,?

(Sufficiente dire quale e perché senza produrre un controesempio.)
\end{exercise}









\end{document}


