\documentclass[10pt]{article}
\usepackage[utf8]{inputenc}
\usepackage[a4paper,height=24cm,width=13cm]{geometry}
\usepackage[italian]{babel}
\usepackage{amssymb}
\usepackage{dsfont}
\usepackage{calc}
\usepackage{graphicx}
\usepackage{pstricks}
\usepackage{pst-node}
\usepackage{fourier}
\usepackage{euscript}
\usepackage{amsmath,amssymb, amsthm}

\def\lh{\textrm{lh}}
\def\phi{\varphi}
\def\P{\EuScript P}
\def\M{\EuScript M}
\def\D{\EuScript D}
\def\U{\EuScript U}
\def\S{\EuScript S}
\def\sm{\smallsetminus}
\def\niff{\nleftrightarrow}
\def\ZZ{\mathds Z}
\def\NN{\mathds N}
\def\QQ{\mathds Q}
\def\RR{\mathds R}
\def\<{\langle}
\def\>{\rangle}
\def\E{\exists}
\def\A{\forall}
\def\0{\varnothing}
\def\imp{\rightarrow}
\def\iff{\leftrightarrow}
\def\IMP{\Rightarrow}
\def\IFF{\Leftrightarrow}
\def\range{\textrm{im}}
\def\Mod{\textrm{Mod}}
\def\Aut{\textrm{Aut}}
\def\Th{\textrm{Th}}
\def\acl{\textrm{acl}}
\def\eq{{\rm eq}}
\def\tp{\textrm{tp}}
\def\equivL{\stackrel{\smash{\scalebox{.5}{\rm L}}}{\equiv}}

\newcommand{\labella}[1]{{\sf\footnotesize #1}\hfill}
\renewenvironment{itemize}
  {\begin{list}{$\triangleright$}{%
   \setlength{\parskip}{0mm}
   \setlength{\topsep}{0mm}
   \setlength{\rightmargin}{0mm}
   \setlength{\listparindent}{0mm}
   \setlength{\itemindent}{0mm}
   \setlength{\labelwidth}{3ex}
   \setlength{\itemsep}{0mm}
   \setlength{\parsep}{0mm}
   \setlength{\partopsep}{0mm}
   \setlength{\labelsep}{1ex}
   \setlength{\leftmargin}{\labelwidth+\labelsep}
   \let\makelabel\labella}}{%
   \end{list}}
%\def\ssf#1{\textsf{\small #1}}
\newcounter{ex}
\newenvironment{exercise}{\clearpage\addtocounter{ex}{1}\textbf{Esercizio \theex.\quad}}{}
\pagestyle{empty}
\parindent0ex
\parskip2ex
\raggedbottom
\def\nsR{{}^*\!\RR}
\def\ssf#1{\textsf{#1}}
\renewcommand{\baselinestretch}{1.3}


\usepackage{fancyhdr}
\pagestyle{fancy}
\lhead{Logica Matematica a.a.~2016/17\hfill Nome Cognome}
\rhead{}
\cfoot{}
\rfoot{\rput(1.5,-0.5){\small\thepage}}

\begin{document}


\clearpage%%%%%%%%%%%%%%%%%%%%%%%%%%%%%%%
\rhead{}\setcounter{ex}{0}


\begin{exercise}
Let $\Phi\subseteq L$ be a set of sentences and suppose that $\vdash\psi\iff\bigvee\Phi$ for some sentence $\psi$. Prove that there is a finite $\Phi_0\subseteq\Phi$ such that  $\vdash\psi\iff\bigvee\Phi_0$.

Nota. Anche se $\bigvee\Phi$ non è una formula del prim'ordine la semantica è quella naturale: $M\models\bigvee\Phi$ se qualche formula in $\Phi$ è vera in $M$.
\end{exercise}

\begin{exercise}\label{2}
Let $a,b,c\in N\models T_{\rm rg}$. Prove that $r(a,N)=r(b,N)\cap r(c,N)$ occurs only in the trivial case $a=b=c$.
\end{exercise}

\begin{exercise}
Prove that if $M$ is either the infinite complete graph, the infinite empty graph, or the random graph and $M_1,M_2\subseteq M$ are such that $M_1\sqcup M_2=M$, then $M_1\simeq M$ or $M_2\simeq M$.
\end{exercise}

\begin{exercise}
Prove that for every $b\in N\models T_{\rm rg}$ the set $r(b,N)$ is a random graph. You may assume for simplicity that $N$ is countable. Is every random graph $M\subseteq N$ of the form $r(b,N)$ for some $b\in N$~? Suggerimento: per la seconda domanda si usi l'esercizio precedente. Alternativamente si può usare l'esercizio~\ref{2}.
\end{exercise}
\end{document}


