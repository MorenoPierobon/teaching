\documentclass[10pt]{article}
\usepackage[utf8]{inputenc}
\usepackage[a4paper,height=24cm,width=13cm]{geometry}
\usepackage[italian]{babel}
\usepackage{amssymb}
\usepackage{dsfont}
\usepackage{calc}
\usepackage{graphicx}
\usepackage{pstricks}
\usepackage{pst-node}
\usepackage{fourier}
\usepackage{euscript}
\usepackage{amsmath,amssymb, amsthm}

\def\lh{\textrm{lh}}
\def\phi{\varphi}
\def\P{\EuScript P}
\def\M{\EuScript M}
\def\D{\EuScript D}
\def\U{\EuScript U}
\def\S{\EuScript S}
\def\sm{\smallsetminus}
\def\niff{\nleftrightarrow}
\def\ZZ{\mathds Z}
\def\NN{\mathds N}
\def\PP{\mathds P}
\def\QQ{\mathds Q}
\def\RR{\mathds R}
\def\<{\langle}
\def\>{\rangle}
\def\E{\exists}
\def\A{\forall}
\def\0{\varnothing}
\def\imp{\rightarrow}
\def\iff{\leftrightarrow}
\def\IMP{\Rightarrow}
\def\IFF{\Leftrightarrow}
\def\range{\textrm{im}}
\def\Mod{\textrm{Mod}}
\def\Aut{\textrm{Aut}}
\def\Th{\textrm{Th}}
\def\acl{\textrm{acl}}
\def\eq{{\rm eq}}
\def\tp{\textrm{tp}}
\def\equivL{\stackrel{\smash{\scalebox{.5}{\rm L}}}{\equiv}}
\def\swedge{\mathbin{\raisebox{.2ex}{\tiny$\mathbin\wedge$}}}
\def\svee{\mathbin{\raisebox{.2ex}{\tiny$\mathbin\vee$}}}

\newcommand{\labella}[1]{{\sf\footnotesize #1}\hfill}
\renewenvironment{itemize}
  {\begin{list}{$\triangleright$}{%
   \setlength{\parskip}{0mm}
   \setlength{\topsep}{0mm}
   \setlength{\rightmargin}{0mm}
   \setlength{\listparindent}{0mm}
   \setlength{\itemindent}{0mm}
   \setlength{\labelwidth}{3ex}
   \setlength{\itemsep}{0mm}
   \setlength{\parsep}{0mm}
   \setlength{\partopsep}{0mm}
   \setlength{\labelsep}{1ex}
   \setlength{\leftmargin}{\labelwidth+\labelsep}
   \let\makelabel\labella}}{%
   \end{list}}
%\def\ssf#1{\textsf{\small #1}}
\newcounter{ex}
\newenvironment{exercise}{\clearpage\addtocounter{ex}{1}\textbf{Esercizio \theex.\quad}}{}
\pagestyle{empty}
\parindent0ex
\parskip2ex
\raggedbottom
\def\nsR{{}^*\!\RR}
\def\ssf#1{\textsf{#1}}
\renewcommand{\baselinestretch}{1.3}


\usepackage{fancyhdr}
\pagestyle{fancy}
\lhead{Logica Matematica a.a.~2016/17}
\rhead{}
\cfoot{}
\rfoot{\rput(1.5,-0.5){\small\thepage}}

\begin{document}

\clearpage%%%%%%%%%%%%%%%%%%%%%%%%%%%%%%%
\rhead{\hfill Stefano Silvestrini}\setcounter{ex}{0}

\begin{exercise}
Let $p(x)\subseteq L(A)$, with $|x|<\omega$. Prove that if $p(\U)$ is infinite then it has cardinality $\kappa$. Show that this may not be true for all $p(x)\subseteq L(\U)$. 
\end{exercise}


\begin{exercise} 
Let $\phi(x,y)\in L(\U)$. Prove that the following are equivalent
\begin{itemize}
\item[1.] there is a sequence $\<a_i\,:\,i\in\omega\>$ such that $\phi(\U,a_i)\subset\phi(\U,a_{i+1})$ for every $i<\omega$;
\item[2.] there is a sequence $\<a_i\,:\,i\in\omega\>$ such that $\phi(\U,a_{i+1})\subset\phi(\U,a_i)$ for every $i<\omega$.  
\end{itemize}
\end{exercise}



\begin{exercise}
Let $\phi(x\,;z)\in L$. Prove that if the set $\big\{\phi(a\,;\U)\ :\ a\in\U^{|x|}\big\}$ is infinite then it has cardinality $\kappa$.
\end{exercise}


\clearpage%%%%%%%%%%%%%%%%%%%%%%%%%%%%%%%
\rhead{\hfill Daniel Marini}\setcounter{ex}{0}

\begin{exercise}
Let $a\in\U$ be such that ${\EuScript O}(a/A)=\{a\}$. Prove that there is a formula $\phi(x)\in L(A)$ such that $\phi(a)\wedge\E^{=1}x\;\phi(x)$.

Suggerimento: si ricordi che ${\EuScript O}(a/A)=p(\U)$ per $p(x)=\tp(a/A)$.
\end{exercise}


\begin{exercise}
Let $\phi(x)\subseteq L(\U)$ be such that $\big\{f[\phi(\U)] \ :\ f\in\Aut(\U/A)\big\}$ contains exactly $2$ elements.  Prove $\phi(\U)$ is definable over \textit{any\/} model $M$ containing $A$.
\end{exercise}


\begin{exercise}
Let $p(x)\subseteq L(A)$, with $|x|<\omega$. Prove that if $p(\U)$ is infinite then it has cardinality $\kappa$. Show that this may not be true if $x$ is an infinite tuple. 

Suggerimento: è sufficiente ci sia un insieme definibile con esattamente due elementi.
\end{exercise}

\clearpage%%%%%%%%%%%%%%%%%%%%%%%%%%%%%%%
\rhead{\hfill Lucia Giromini}\setcounter{ex}{0}

\begin{exercise}
Let $p(x)\subseteq L$ be such that $p(\U)$ contains just one element. Prove that there is a formula $\phi(x)$, a conjunctions of formulas in $p(x)$, such that $p(\U)=\phi(\U)$.
\end{exercise}

\begin{exercise}
Let $p(x)\subseteq L(A)$ be such that $\neg p(x)\iff q(x)$ for some $q(x)\subseteq L(B)$. Prove that $p(x)$ is equivalent to some conjunction of formulas in $p(x)$.
\end{exercise}


\begin{exercise} 
Let $\phi(x,y)\in L(\U)$. Prove that the following are equivalent
\begin{itemize}
\item[1.] there is a sequence $\<a_i\,:\,i\in\omega\>$ such that $\phi(\U,a_i)\subset\phi(\U,a_{i+1})$ for every $i<\omega$;
\item[2.] there is a sequence $\<a_i\,:\,i\in\omega\>$ such that $\phi(\U,a_{i+1})\subset\phi(\U,a_i)$ for every $i<\omega$.  
\end{itemize}
\end{exercise}

\clearpage%%%%%%%%%%%%%%%%%%%%%%%%%%%%%%%
\rhead{\hfill Chiara Neri}\setcounter{ex}{0}

\begin{exercise}
Let $M$ and $N$ be elementarily homogeneous structures of the same cardinality $\lambda$. Suppose that $M\models\E x\, p(x)\,\IFF\,N\models\E x\, p(x)$ for every $p(x)\subseteq L$ such that $|x|<\lambda$. Prove that the two structures are isomorphic.

Suggerimento: vedi dimostrazione di \textit{universale $+$ homogeneo $\IMP$ ricco}
\end{exercise}


\begin{exercise}
Let $\phi(x\,;z)\in L$. Prove that if the set $\big\{\phi(a\,;\U)\ :\ a\in\U^{|x|}\big\}$ is infinite then it has cardinality $\kappa$. Does the claim remains true with a type $p(x\,;z)\subseteq L$ for $\phi(x\,;z)$?

Suggerimento: potrebbe esserci un controesempio in $\U\equiv\NN$ nel linguaggio degli ordini.
\end{exercise}


\begin{exercise}
Let $\phi(x)\subseteq L(\U)$ be such that $\big\{f[\phi(\U)] \ :\ f\in\Aut(\U/A)\big\}$ contains exactly $2$ elements.  Prove $\phi(\U)$ is definable over \textit{any\/} model $M$ containing $A$.
\end{exercise}

\end{document}


