\documentclass[10pt]{article}
\usepackage[a4paper,height=24cm,width=14cm]{geometry}
\usepackage[italian]{babel}
\usepackage{amssymb}
\usepackage{dsfont}
\usepackage{calc}
\usepackage{graphicx}
\usepackage{pstricks}
\usepackage{pst-node}
\usepackage{fourier}
\usepackage{euscript}
\usepackage{amsmath,amssymb, amsthm}

\def\lh{\textrm{lh}}
\def\phi{\varphi}
\def\P{\EuScript P}
\def\M{\EuScript M}
\def\D{\EuScript D}
\def\U{\EuScript U}
\def\S{\EuScript S}
\def\sm{\smallsetminus}
\def\niff{\nleftrightarrow}
\def\ZZ{\mathds Z}
\def\NN{\mathds N}
\def\QQ{\mathds Q}
\def\RR{\mathds R}
\def\<{\langle}
\def\>{\rangle}
\def\E{\exists}
\def\A{\forall}
\def\0{\varnothing}
\def\imp{\rightarrow}
\def\iff{\leftrightarrow}
\def\IMP{\Rightarrow}
\def\IFF{\Leftrightarrow}
\def\range{\textrm{im}}
\def\Mod{\textrm{Mod}}
\def\Aut{\textrm{Aut}}
\def\Th{\textrm{Th}}
\def\acl{\textrm{acl}}
\def\eq{{\rm eq}}
\def\tp{\textrm{tp}}
\def\equivL{\stackrel{\smash{\scalebox{.5}{\rm L}}}{\equiv}}

\newcommand{\labella}[1]{{\sf\footnotesize #1}\hfill}
\renewenvironment{itemize}
  {\begin{list}{$\triangleright$}{%
   \setlength{\parskip}{0mm}
   \setlength{\topsep}{0mm}
   \setlength{\rightmargin}{0mm}
   \setlength{\listparindent}{0mm}
   \setlength{\itemindent}{0mm}
   \setlength{\labelwidth}{3ex}
   \setlength{\itemsep}{0mm}
   \setlength{\parsep}{0mm}
   \setlength{\partopsep}{0mm}
   \setlength{\labelsep}{1ex}
   \setlength{\leftmargin}{\labelwidth+\labelsep}
   \let\makelabel\labella}}{%
   \end{list}}
%\def\ssf#1{\textsf{\small #1}}
\newcounter{ex}
\newenvironment{exercise}{\clearpage\addtocounter{ex}{1}\textbf{Esercizio \theex.\quad}}{}
\pagestyle{empty}
\parindent0ex
\parskip2ex
\raggedbottom
\def\nsR{{}^*\!\RR}
\def\ssf#1{\textsf{#1}}
\renewcommand{\baselinestretch}{1.3}


\usepackage{fancyhdr}
\pagestyle{fancy}
\lhead{Logica Matematica a.a.~2016/17\hfill Nome Cognome}
\rhead{}
\cfoot{}
\rfoot{\rput(1.5,-0.5){\small\thepage}}

\begin{document}


\clearpage%%%%%%%%%%%%%%%%%%%%%%%%%%%%%%%
\rhead{}\setcounter{ex}{0}


\begin{exercise}
Prove that if $T$ has exactly $2$ maximal consistent extension $T_1$ and $T_2$ then there is a sentence $\phi$ such that $T,\phi\vdash T_1$ and $T,\neg\phi\vdash T_2$.
\end{exercise} 

\begin{exercise}\label{ex_LS}
Assume $L$ is countable and let $M\subseteq N$ and $A\subseteq N$ be both countable. Prove that there is a countable model $K$ such that $A\subseteq K\preceq N$ and $K\cap M\preceq N$ (in particular, $K\cap M$ is a model).  

Hint 1: adapt the construction used to prove the downward L\"owenheim-Skolem.

Hint 2 (alternative construction): construct two countable chains of countable models such that $K_i\cap M\subseteq M_i\preceq M$ and $A\cup M_i\subseteq K_{i+1}\preceq M$. The required model is $K=\bigcup_{i\in\omega}K_i$ as $K\cap M=\bigcup_{i\in\omega}M_i$.

\end{exercise}


\begin{exercise}\label{VaughtEsempio3modelli}
Let $L$ be the language of strict orders augmented with countably many constants $\big\{c_i: i\in\omega\big\}$. Let $T$ be the theory that extends $T_{\rm dlo}$ with the axioms $c_i<c_{i+1}$ for all $i$. Exhibit three non isomorphic models of this theory. Say if an extension lemma similar to 4.1 holds with $N$ one of these models and any model of $T$ as $M$.
\end{exercise}


\end{document}


