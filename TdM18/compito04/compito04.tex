\documentclass[10pt]{article}
\usepackage[utf8]{inputenc}
\usepackage[a4paper,height=24cm,width=13cm]{geometry}
\usepackage[italian]{babel}
\usepackage{amssymb}
\usepackage{dsfont}
\usepackage{calc}
\usepackage{graphicx}
\usepackage{pstricks}
\usepackage{pst-node}
\usepackage{fourier}
\usepackage{euscript}
\usepackage{amsmath,amssymb, amsthm}

\def\lh{\textrm{lh}}
\def\phi{\varphi}
\def\P{\EuScript P}
\def\M{\EuScript M}
\def\D{\EuScript D}
\def\U{\EuScript U}
\def\S{\EuScript S}
\def\sm{\smallsetminus}
\def\niff{\nleftrightarrow}
\def\ZZ{\mathds Z}
\def\NN{\mathds N}
\def\PP{\mathds P}
\def\QQ{\mathds Q}
\def\RR{\mathds R}
\def\<{\langle}
\def\>{\rangle}
\def\E{\exists}
\def\A{\forall}
\def\0{\varnothing}
\def\imp{\rightarrow}
\def\iff{\leftrightarrow}
\def\IMP{\Rightarrow}
\def\IFF{\Leftrightarrow}
\def\range{\textrm{im}}
\def\Mod{\textrm{Mod}}
\def\Aut{\textrm{Aut}}
\def\Th{\textrm{Th}}
\def\acl{\textrm{acl}}
\def\eq{{\rm eq}}
\def\tp{\textrm{tp}}
\def\equivL{\stackrel{\smash{\scalebox{.5}{\rm L}}}{\equiv}}
\def\swedge{\mathbin{\raisebox{.2ex}{\tiny$\mathbin\wedge$}}}
\def\svee{\mathbin{\raisebox{.2ex}{\tiny$\mathbin\vee$}}}

\newcommand{\labella}[1]{{\sf\footnotesize #1}\hfill}
\renewenvironment{itemize}
  {\begin{list}{$\triangleright$}{%
   \setlength{\parskip}{0mm}
   \setlength{\topsep}{0mm}
   \setlength{\rightmargin}{0mm}
   \setlength{\listparindent}{0mm}
   \setlength{\itemindent}{0mm}
   \setlength{\labelwidth}{3ex}
   \setlength{\itemsep}{0mm}
   \setlength{\parsep}{0mm}
   \setlength{\partopsep}{0mm}
   \setlength{\labelsep}{1ex}
   \setlength{\leftmargin}{\labelwidth+\labelsep}
   \let\makelabel\labella}}{%
   \end{list}}
%\def\ssf#1{\textsf{\small #1}}
\newcounter{ex}
\newenvironment{exercise}{\clearpage\addtocounter{ex}{1}\textbf{Esercizio \theex.\quad}}{}
\pagestyle{empty}
\parindent0ex
\parskip2ex
\raggedbottom
\def\nsR{{}^*\!\RR}
\def\ssf#1{\textsf{#1}}
\renewcommand{\baselinestretch}{1.3}


\usepackage{fancyhdr}
\pagestyle{fancy}
\lhead{Logica Matematica a.a.~2017/18}
\rhead{}
\cfoot{}
\rfoot{\rput(1.5,-0.5){\small\thepage}}

\begin{document}

\clearpage%%%%%%%%%%%%%%%%%%%%%%%%%%%%%%%
\rhead{\hfill }\setcounter{ex}{0}

\begin{exercise}\label{ex_feqthm_senza_eq}
Let $A\subseteq\U$, let $\D$ be a definable set with finite orbit over $A$. Without using $\U^\eq$ prove that $\D$ is union of classes of a finite equivalence relation definable over $A$.

L'esercizio può richiedere un po' di riflessione (e discussione). I seguenti due esercizi sono in alternativa.
\end{exercise}


\begin{exercise}
Let $p(x)\subseteq L(A)$ and let $\phi(x\,;y)\in L(A)$ be a formula that defines, when restricted to $p(\U)$, an equivalence relation with finitely many classes. Prove that there is a finite equivalence relation definable over $A$ that coincides with $\phi(x\,;y)$ on $p(\U)$.
\end{exercise}


\begin{exercise}
Let $T$ be strongly minimal and let $\phi(x\,;y)\in L(A)$ with $|x|=|y|=1$. For arbitrary $c\in\U$, prove that if the orbit of $\phi(\U\,; c)$ over $A$ is finite, then $\phi(\U\,; c)$ is definable over $\acl A$.
\end{exercise}




\end{document}


