\documentclass[10pt]{article}
\usepackage[utf8]{inputenc}
\usepackage[a4paper,height=24cm,width=13cm]{geometry}
\usepackage[italian]{babel}
\usepackage{amssymb}
\usepackage{dsfont}
\usepackage{calc}
\usepackage{graphicx}
\usepackage{pstricks}
\usepackage{pst-node}
\usepackage{fourier}
\usepackage{euscript}
\usepackage{amsmath,amssymb, amsthm}

\def\lh{\textrm{lh}}
\def\phi{\varphi}
\def\P{\EuScript P}
\def\M{\EuScript M}
\def\D{\EuScript D}
\def\U{\EuScript U}
\def\S{\EuScript S}
\def\sm{\smallsetminus}
\def\niff{\nleftrightarrow}
\def\ZZ{\mathds Z}
\def\NN{\mathds N}
\def\PP{\mathds P}
\def\QQ{\mathds Q}
\def\RR{\mathds R}
\def\<{\langle}
\def\>{\rangle}
\def\E{\exists}
\def\A{\forall}
\def\0{\varnothing}
\def\imp{\rightarrow}
\def\iff{\leftrightarrow}
\def\IMP{\Rightarrow}
\def\IFF{\Leftrightarrow}
\def\range{\textrm{im}}
\def\Mod{\textrm{Mod}}
\def\Aut{\textrm{Aut}}
\def\Th{\textrm{Th}}
\def\acl{\textrm{acl}}
\def\eq{{\rm eq}}
\def\tp{\textrm{tp}}
\def\equivL{\stackrel{\smash{\scalebox{.5}{\rm L}}}{\equiv}}
\def\swedge{\mathbin{\raisebox{.2ex}{\tiny$\mathbin\wedge$}}}
\def\svee{\mathbin{\raisebox{.2ex}{\tiny$\mathbin\vee$}}}

\newcommand{\labella}[1]{{\sf\footnotesize #1}\hfill}
\renewenvironment{itemize}
  {\begin{list}{$\triangleright$}{%
   \setlength{\parskip}{0mm}
   \setlength{\topsep}{0mm}
   \setlength{\rightmargin}{0mm}
   \setlength{\listparindent}{0mm}
   \setlength{\itemindent}{0mm}
   \setlength{\labelwidth}{3ex}
   \setlength{\itemsep}{0mm}
   \setlength{\parsep}{0mm}
   \setlength{\partopsep}{0mm}
   \setlength{\labelsep}{1ex}
   \setlength{\leftmargin}{\labelwidth+\labelsep}
   \let\makelabel\labella}}{%
   \end{list}}
%\def\ssf#1{\textsf{\small #1}}
\newcounter{ex}
\newenvironment{exercise}{\par\bigskip\addtocounter{ex}{1}\textbf{Exercise \theex.\quad}}{}
\newenvironment{hint}{\par\smallskip\color{blue}\textbf{Hint.\quad}}{}
\newenvironment{remark}{\par\smallskip\color{blue}\textbf{Remark.\quad}}{}
\pagestyle{empty}
\parindent0ex
\parskip1ex
\raggedbottom
\def\nsR{{}^*\!\RR}
\def\ssf#1{\textsf{#1}}
\renewcommand{\baselinestretch}{1.2}


\usepackage{fancyhdr}
\pagestyle{fancy}
\lhead{Self-diagnostic test to access the prerequisites for the course ???}
\rhead{}
\cfoot{}
%\rfoot{\rput(1.5,-0.5){\small\thepage}}

\begin{document}

\clearpage%%%%%%%%%%%%%%%%%%%%%%%%%%%%%%%
\rhead{\hfill }\setcounter{ex}{0}

{\color{blue}
\textbf{Prerequisites}
\begin{itemize}
\item Syntax and semantics of first-order languages.
\item Elementary equivalence and elementary substructures.
\item Isomorphisms and embeddings between structures.
\end{itemize}}


\begin{exercise}
Let $M$ be an $L$-structure and let $\psi(x), \phi(x,y)\in L$. For each of the following conditions, write a sentence true in $M$ exactly when the condition holds: 
\begin{itemize}
\item[a.] $\psi(M)\ \in\ \big\{\phi(a,M): a\in M\big\}$;
\item[b.] $\big\{\phi(a,M): a\in M\big\}$ contains at least two elements;
\item[c.] $\big\{\phi(a,M): a\in M\big\}$ contains only sets that are pairwise disjoint.
\end{itemize}
\begin{remark}
Recall that $\phi(a,M)=\big\{b\ :\ M\models\phi(a,b)\big\}$. Do not confuse definable subsets of $M$ with \textit{sets\/} of definable subsets of $M$. Beware that the relation between parameters and the sets they define is not one-to-one.
\end{remark}
\end{exercise}

\begin{exercise}
Let $M$ be a structure in a signature with only a binary relation symbol $r$. Write a sentence $\phi$ such that, for every structure $M$,
\begin{itemize}
\item[a.] $M\models\phi$ if and only if there is an $A\subseteq M$ such that $r^M\ \subseteq\ A\times\neg A$.
\end{itemize}
\begin{remark}
Note that $\phi$ asserts that $r^M$ is a bipartite directed graph. As motivation, it is interesting to note that \ssf{a} is an asymmetric version of 
\begin{itemize}
\item[b.] $M\models\psi$ if and only if there is an $A\subseteq M$ such that $r^M\ \subseteq\ (A\times \neg A)\;\cup\;(\neg A\times A)$.
\end{itemize}
The sentence $\psi$ says that the (undirected) graph naturally associated to $r^M$ is a \textit{bipartite graph}, or equivalently that its \textit{chromatic number is \ $2$} (its vertices are colorable with $2$ colors so that any two adjacent vertices have different colors). The compactness theorem can be used to prove that there is no such sentence $\psi$ (not required).
\end{remark}
\end{exercise}

\begin{exercise}\label{ex_finite_nature}
Prove that $M\equiv_AN$ if and only if $M\equiv_BN$ for every finite $B\subseteq A$.
\begin{remark}
This is called the finite nature (or character) of elementary equivalence.
\end{remark}
\end{exercise}


\begin{exercise}
Prove that in the language of orders $(0,1)\not\preceq(0,2]$ and $(0,1)\preceq(0,2)$.
\begin{remark}
Assume the intervals above are all rational, or all real. Give a short proof without assuming elimination of quantifiers. Exercise~3 above may be useful.
\end{remark}
\end{exercise}

\begin{exercise}
Let $M\preceq N$ and let $\phi(x,z)\in L$. Suppose there are finitely many sets of the form $\phi(a,N)$ for some $a\in N^{|x|}$. Prove that all these sets are definable over $M$.
\begin{hint}
Exercise~1 above may be useful.
\end{hint}
\end{exercise}

\begin{exercise}
Prove that if $A_1,\dots,A_n\subseteq\big\{1,\dots,m\big\}$ are distinct sets such that $A_i\cap A_j\neq\0$ for all $i, j$, then $n\le 2^{m-1}$.
\begin{hint}
The proof is two lines.
\end{hint}
\begin{remark}
This exercise is not directly related to model theory. However, basic combinatorial principles are pivotal in model theory, as they are used to distinguish first-order structures according to their complexity.

(P\'eter Frankl credits this exercise to Katona who used it to advertise his combinatorics seminar by saying: ``if you can solve this exercise, come to the seminar''.)
\end{remark}
\end{exercise}

\end{document}


